\documentclass{article}
\usepackage{template}
\usepackage{changepage} % Modify page width

\geometry{
	a4paper,
	margin = 10mm
}

\pagenumbering{gobble}

\begin{document}
\noindent A \textbf{vector space} $V$ is closed under vector addition 
and scalar multiplication: $\symbfit{u}+\symbfit{v} \in V$ and 
$k\symbfit{u} \in V$.

\noindent A \textit{subset} $W$ of a vector space $V$ is called a 
\textbf{subspace} of $V$ if $W$ is itself a vector space. The 
intersection of subspaces is also a subspace of $V$.

\noindent $S$ is \textbf{linearly independent (LI)} if 
$k_1 \symbfit{v}_1 + k_2 \symbfit{v}_2 + \cdots + k_n \symbfit{v}_n = \symbfup{0}$ 
has $k_i=0$. $S$ forms a \textbf{basis} for $V$ if $S$ spans $V$ and 
$S$ is LI.

\noindent For $\symbfit{A}\in\mathbb{R}^{m \times n}$: 
$r = \rank{\symbfit{A}} = \rank{\symbfit{A}^\top} = \dim{\left( \columnspace{A} \right)} = \dim{\left( \rowspace{A} \right)}$. 
$\vnull{\left( \symbfit{A} \right)} = \dim{\left( \nullspace{A} \right)} = n-r$ 
and $\vnull{\left( \symbfit{A}^\top \right)} = \dim{\left( \leftnullspace{A} \right)} = m-r$.

\noindent The \textbf{four fundamental subspaces}: 
$\left( \columnspace{A} \right)^\perp = \leftnullspace{A}$ and 
$\left( \rowspace{A} \right)^\perp = \nullspace{A}$.

\noindent \textbf{Row equivalent} matrices have the same 
\underline{row space} and \underline{null space}. 

\noindent The subspaces $U$ and $W$ of a vector space $V$ are 
\textbf{orthogonal subspaces} iff 
$\forall \symbfit{u}\in U:\forall \symbfit{w}\in W:\symbfit{u}^{\top}\symbfit{w} = 0$. 
$\symbfit{v}^\top \symbfit{v}=\norm{\symbfit{v}}^2$. 

\noindent The \textbf{orthogonal complement} of 
$U$: $U^{\perp} = \left\{ \forall \symbfit{u}\in U:\symbfit{v}\in V: 
\symbfit{v}^{\top}\symbfit{u}=0 \right\}$ and 
$\left( U^{\perp} \right)^{\perp} = U$. 
$\dim{U} + \dim{U^{\perp}} = \dim{V}$.

\noindent \textbf{Projections}: $\proj_{\symbfit{a}}\symbfit{b} 
= \symbfit{a} x 
= \symbfit{a} \frac{\symbfit{a}^\top \symbfit{b}}{\symbfit{a}^\top \symbfit{a}}$. 
$\forall \symbfit{w}\in W:\symbfit{w}\neq \symbfit{p}:\proj_{W}\symbfit{b} 
= \symbfit{A}\symbfit{\hat{x}} 
= \symbfit{A}\left( \symbfit{A}^\top \symbfit{A} \right)^{-1}\symbfit{A}^\top \symbfit{b}
:\norm{\symbfit{b}-\symbfit{p}}<\norm{\symbfit{b}-\symbfit{w}}$.

\noindent $\det{\left( \symbfit{A} \right)} = \sum_{j=1}^n a_{ij}C_{ij} = \sum_{i=1}^n a_{ij}C_{ij}$, 
where $C_{ij}=\left( -1 \right)^{i+j}M_{ij}$. 
$\symbfit{A}^{-1}=\frac{1}{\det{\symbfit{A}}} \adj{\left( \symbfit{A} \right)}$, 
where $\adj{\left( \symbfit{A} \right)}=C^\top$.

\noindent \textbf{Linear transformations}: 
$T:V\rightarrow W \iff \forall \symbfit{u},\: \symbfit{v} \in V:\forall k \in \mathbb{R}:T\left(\symbfit{u}+\symbfit{v}\right) = T\left(\symbfit{u}\right) + T\left(\symbfit{v}\right) \wedge T\left(k\symbfit{u}\right) = kT\left(\symbfit{u}\right)$. 

\noindent \textbf{Rotations}: 
$\symbfit{R}_x = \mqty[1 & 0 & 0 \\ 0 & \cos{\left( \theta \right)} & -\sin{\left( \theta \right)} \\ 0 & \sin{\left( \theta \right)} & \cos{\left( \theta \right)}]$. 
$\symbfit{R}_y = \mqty[\cos{\left( \theta \right)} & 0 & \sin{\left( \theta \right)} \\ 0 & 1 & 0 \\ -\sin{\left( \theta \right)} & 0 & \cos{\left( \theta \right)}]$. 
$\symbfit{R}_z = \mqty[\cos{\left( \theta \right)} & -\sin{\left( \theta \right)} & 0 \\ \sin{\left( \theta \right)} & \cos{\left( \theta \right)} & 0 \\ 0 & 0 & 1]$. 
Anticlockwise rotations looking down from the positive direction of the 
axis of rotation.
In 2-d: 
$\symbfit{R} = \mqty[\cos{\left( \theta \right)} & -\sin{\left( \theta \right)} \\ \sin{\left( \theta \right)} & \cos{\left( \theta \right)}]$.
		
\noindent \textbf{Shears}: 
$\symbfit{S}_x = \mqty[1 & a & b \\ 0 & 1 & 0 \\ 0 & 0 & 1]$. 
$\symbfit{S}_y=\mqty[1 & 0 & 0 \\ a & 1 & b \\ 0 & 0 & 1]$. 
$\symbfit{S}_z=\mqty[1 & 0 & 0 \\ 0 & 1 & 0 \\ a & b & 1]$. 
Where the standard basis vector in the subscripted axis maps to itself. 
Think about where the standard basis vectors maps.

\noindent \textbf{Reflections}: 
$\symbfit{M}_{xy}=\mqty[1 & 0 & 0 \\ 0 & 1 & 0 \\ 0 & 0 & -1]$. 
$\symbfit{M}_{xz}=\mqty[1 & 0 & 0 \\ 0 & -1 & 0 \\ 0 & 0 & 1]$. 
$\symbfit{M}_{yz}=\mqty[-1 & 0 & 0 \\ 0 & 1 & 0 \\ 0 & 0 & 1]$. 
Where vectors are reflected across the plane formed by the subscripts of 
$\symbfit{M}$.

\noindent \textbf{2-d Reflections} about $y=mx + c$, where $\theta=\arctan{\left( m \right)}$:
\begin{align*}
	T\left( \symbfit{v} \right) &= \symbfit{R} \symbfit{M}_{xz} \symbfit{R}^{-1} \left( \symbfit{v} - \mqty[0 \\ c] \right) + \mqty[0 \\ c] \\
	&= \mqty[\cos{\left( \theta \right)} & -\sin{\left( \theta \right)} \\ \sin{\left( \theta \right)} & \cos{\left( \theta \right)}] \mqty[1 & 0 \\ 0 & -1] \mqty[\cos{\left( \theta \right)} & -\sin{\left( \theta \right)} \\ \sin{\left( \theta \right)} & \cos{\left( \theta \right)}]^{-1} \left( \symbfit{v} - \mqty[0 \\ c] \right) + \mqty[0 \\ c] \\
	&= \frac{1}{1+m^2}\mqty[1-m^2 & 2m \\ 2m & m^2-1] \left( \symbfit{v} - \mqty[0 \\ c] \right) + \mqty[0 \\ c]
\end{align*}

\noindent \textbf{Invariant (IV) subspaces}: 
For $T:V\rightarrow V$. $\mathcal{V}$ is IV if 
$T\left(\mathcal{V}\right)\subseteq \mathcal{V}\iff\forall \symbfit{v}
\in \mathcal{V}\implies T\left(\symbfit{v}\right)\in \mathcal{V}$. 
\textbf{Trivial IV subspaces}: $V$, 
$\vim{\left( T \right)} = T(V) = \left\{ T\left(\symbfit{v}\right) : 
\symbfit{v}\in V \right\} \subset W$, 
$\vker{\left( T \right)} = \left\{ \symbfit{v}\in V : 
T\left(\symbfit{v}\right) = \symbfup{0} \right\}$, 
$\left\{ \symbfup{0} \right\}$, and any linear combination of IVs.

\noindent \textbf{Eigenspace} (1-d IV subspace): 
$\mathcal{V} = \left\{ \forall \symbfit{q}\in \mathcal{V}:\exists 
\lambda \in \mathbb{C}:T\left(\symbfit{q}\right) = \lambda \symbfit{q} \right\}$ 
where $\lambda_i$ are the eigenvalues of $\symbfit{A}$ and 
$\symbfit{q}_i$ are the eigenvectors of $\symbfit{A}$. 
$\left( \symbfit{A} - \lambda \mathbb{1} \right) \symbfit{q}=\symbfup{0}$. 
If $\symbfit{A}$ is invertible: 
$\det{\left( \symbfit{A} - \lambda\mathbb{1} \right)} = \symbfup{0}$. 
\textbf{Characteristic polynomial}: 
$p_n\left(\lambda\right) = \det{\left( \symbfit{A}_n - \lambda\mathbb{1}_n \right)}$; 
in 2-d: $p_2\left(\lambda\right) = \lambda^2-\tr{\left( \symbfit{A} \right)}\lambda + \det{\left( \symbfit{A} \right)}$. 
$\tr{\left( \symbfit{A} \right)} = \sum_{i=1}^n \lambda_i$ and 
$\det{\left( \symbfit{A} \right)} = \prod_{i=1}^n \lambda_i$. 

\noindent \textbf{Similarity transformation}: 
$\symbfit{A}\rightarrow \symbfit{Q}^{-1}\symbfit{A}\symbfit{Q}$. If 
$\symbfit{q}_i$ is LI, then $\symbfit{A}$ is 
\textbf{diagonalisable}: $\symbfit{\Lambda}=\symbfit{Q}^{-1}\symbfit{A}\symbfit{Q}$. \\
Where $\symbfit{\Lambda}=\mqty[\dmat{\lambda_1,\: \lambda_2,\: \ddots,\: \lambda_n}]$ 
and 
$\symbfit{Q}=\mqty[\vertbar & \vertbar & & \vertbar \\ \symbfit{q}_1 & 
\symbfit{q}_2 & \cdots & \symbfit{q}_n \\ \vertbar & \vertbar & & \vertbar]$.

\noindent If $\symbfit{A}$ is diagonalisable, 
$\forall k \in \mathbb{N}_0:\symbfit{A}^k = \symbfit{Q} \symbfit{\Lambda}^k \symbfit{Q}^{-1}$. 
The eigenvalues of $\symbfit{A}^k$ are the eigenvalues of $\symbfit{A}$ 
to the $k$-th power: $\lambda_1^k,\: \lambda_2^k,\: \dots,\: \lambda_n^k$. 
The eigenvectors of $\symbfit{A}^k$ equal the eigenvectors of $\symbfit{A}$.

\noindent The \textbf{ordinary differential equation (ODE)}: $x' = a x$, 
has the solution: $x(t) = c_1 \e^{a t}$. $c_1$ is determined through 
initial conditions.

\noindent The \textbf{system of differential equations}: $
\left\{
	\setlength\arraycolsep{0pt}
	\begin{array}{ c >{{}}c<{{}} c >{{}}c<{{}} c >{{}}c<{{}} c >{{}}c<{{}} c  }
	x'_1               &=& a_{11}x_1                         &+& a_{12}x_2                         &+& \cdots &+& a_{1n}x_n \\
	x'_2               &=& a_{21}x_1                         &+& a_{22}x_2                         &+& \cdots &+& a_{2n}x_n \\
	\vdotswithin{x'_3} & & \vdotswithin{a_{31}}\phantom{x_1} & & \vdotswithin{a_{32}}\phantom{x_2} & &        & & \vdotswithin{a_{3n}}\phantom{x_n} \\ 
	x'_n               &=& a_{n1}x_1                         &+& a_{n2}x_2                         &+& \cdots &+& a_{nn}x_n 
	\end{array}
\right. \iff 
\dv{t}\mqty[x_1 \\ x_2 \\ \vdots \\ x_n] = \mqty[
	a_{11} & a_{12} & \cdots & a_{1n} \\
	a_{21} & a_{22} & \cdots & a_{2n} \\
	\vdots & \vdots &        & \vdots \\
	a_{n1} & a_{n2} & \cdots & a_{nn}
] \mqty[x_1 \\ x_2 \\ \vdots \\ x_n]$ $\iff \symbfit{x}' = \symbfit{A} \symbfit{x}$ 
can be solved using $\symbfit{x}=\symbfit{Q}\symbfit{u}$, where 
$\symbfit{Q}$ is the matrix that diagonalises $\symbfit{A}$, and 
$\symbfit{u}$ is the solution to 
$\symbfit{u}' = \symbfit{\Lambda} \symbfit{u}$, where 
$\symbfit{\Lambda}$ is the diagonal similarity transformation of 
$\symbfit{A}$.

\noindent If $\symbfit{A}$ is diagonalisable, then for 
$\symbfit{x}' = \symbfit{A} \symbfit{x}$: $\symbfit{x}(t) 
= c_1 \e^{\lambda_1 t} \symbfit{q}_1 + c_2 \e^{\lambda_2 t} \symbfit{q}_2 + 
\cdots + c_n \e^{\lambda_n t} \symbfit{q}_n$.
	
\noindent For the \textbf{higher-order linear differential equation} 
$x^{\left( n \right)} + a_1 x^{\left( n-1 \right)} + \cdots + a_{n-1} x' + a_n x = 0$, 
define: $x_1 = x,\, x_2 = x',\, \dots,\, x_n = x^{\left( n-1 \right)}$, 
and let: $\symbfit{x}=\mqty[x_1 \\ x_2 \\ \cdots \\ x_n]$. Then solve 
the ODE: 
$\dv{t}\mqty[
	x_1 \\
	x_2 \\
	\vdotswithin{x_3} \\
	x_n	
] = \mqty[
	0 & 1 & 0 & \cdots & 0 \\
	0 & 0 & 1 & \cdots & 0 \\
	\vdots & \vdots & \vdots & \ddots & \vdots \\
	0 & 0 & 0 & \cdots & 1 \\
	-a_n & -a_{n-1} & -a_{n-2} & \cdots & -a_1
] \mqty[
	x_1 \\
	x_2 \\
	\vdotswithin{x_3} \\
	x_n	
]$ using diagonalisation.
\newpage

\noindent\textbf{Norm of a vector}: 
$\norm{\symbfit{v}} = \sqrt{v_1^2 + v_2^2+\dots+v_n^2}$.

\noindent\textbf{Unit vector}: 
$\symbfit{\hat{v}} = \frac{\symbfit{v}}{\norm{\symbfit{v}}}$.

\noindent\textbf{Dot product}: 
$\symbfit{v}\cdot\symbfit{w}= v_1 w_1 + v_2 w_2 + \cdots + v_n w_n 
= \norm{\symbfit{v}} \norm{\symbfit{w}} \cos{\left(\theta\right)}$.

\noindent\textbf{Cross product}: $
\symbfit{v}\times\symbfit{w} = 
\mqty|
	\symbfit{\hat{i}} & \symbfit{\hat{j}} & \symbfit{\hat{k}} \\
	v_1 & v_2 & v_3 \\
	w_1 & w_2 & w_3
| =
\norm{\symbfit{v}}\norm{\symbfit{w}}\sin{\left(\theta\right)}\symbfit{\hat{n}}
$.

\noindent\textbf{2-d Inverse}: $\mqty[a & b \\ c & d]^{-1} = \frac{1}{ad - bc} \mqty[d & -b \\ -c & a]$.

\noindent\textbf{Vector space axioms}:

\indent\textbf{Closure under addition}: 
$\symbfit{u}+\symbfit{v} \in V$ 

\indent\textbf{Commutativity of vector addition}: 
$\symbfit{u} + \symbfit{v} = \symbfit{v} + \symbfit{u}$

\indent\textbf{Associativity of vector addition}: 
$\symbfit{u} + \left(\symbfit{v} + \symbfit{w}\right) = 
\left(\symbfit{u} + \symbfit{v}\right) + \symbfit{w}$

\indent\textbf{Additive identity}: 
$\symbfit{u} + \symbfup{0} = \symbfit{u}$

\indent\textbf{Additive inverse}: 
$\symbfit{u} + \left(-\symbfit{u}\right) = \symbfup{0}$

\indent\textbf{Closure under scalar multiplication}: 
$k\symbfit{u} \in V$

\indent\textbf{Distributivity of vector addition}: 
$k \left(\symbfit{u} + \symbfit{v}\right) = k\symbfit{u} + k\symbfit{v}$

\indent\textbf{Distributivity of scalar addition}: 
$\left(k+m\right)\symbfit{u} = k\symbfit{u} + m\symbfit{u}$

\indent\textbf{Associativity of scalar multiplication}: 
$k\left(m\symbfit{u}\right)=\left(km\right)\symbfit{u}$

\indent\textbf{Scalar multiplication identity}: 
$1 \symbfit{u}=\symbfit{u}$

\noindent Subspaces of $\mathbb{R}^2$: $\left\{ \symbfup{0} \right\}$, 
lines through the origin, and $\mathbb{R}^2$.

\noindent Subspaces of $\mathbb{R}^3$: $\left\{ \symbfup{0} \right\}$, 
lines through the origin, planes through the origin, and $\mathbb{R}^3$.

\noindent Subspaces of $\symbfit{M}_{nn}$: Upper triangular matrices, 
lower triangular matrices, diagonal matrices, and $\symbfit{M}_{nn}$.

\noindent\textbf{Determinant properties}:

\indent$\det{\left( \mathbb{1} \right)}=1$. 

\indent Exchanging two rows of a matrix reverses the sign of its 
determinant. 

\indent Determinants are multilinear, so that 
$\mdet{a+a' & b+b' \\ c & d} 
= \mdet{a & b \\ c & d}+\mdet{a' & b' \\ c & d}$ and 
$\mdet{ta & tb \\ c & d}=t\mdet{a & b \\ c & d}$. 

\indent If $\symbfit{A}$ has two equal rows, then 
$\det{\left( \symbfit{A} \right)}=0$. 

\indent Adding a scalar multiple of one row to another does not change 
the determinant of a matrix. 

\indent If $\symbfit{A}$ has a row of zeros, then 
$\det{\left( \symbfit{A} \right)}=0$. 

\indent If $\symbfit{A}$ is triangular, then 
$\det{\left( \symbfit{A} \right)}=\prod_{i=1}^{n} a_{ii}$. 

\indent If $\symbfit{A}$ is singular, then 
$\det{\left( \symbfit{A} \right)}=0$. 

\indent$\det{\left( \symbfit{A}\symbfit{B} \right)} = \det{\left( \symbfit{A} \right)}\det{\left( \symbfit{B} \right)}$. 

\indent$\det{\left( \symbfit{A}^\top \right)} = \det{\left( \symbfit{A} \right)}$ 

\noindent\textbf{Matrix Identities}:

\indent$\symbfit{A}\left( \symbfit{B}\symbfit{C} \right) = \symbfit{A}\symbfit{B}+\symbfit{A}\symbfit{C}$

\indent$\left( \symbfit{A}+\symbfit{B} \right)^\top = \symbfit{A}^\top + \symbfit{B}^\top$

\indent$\left( \symbfit{A}\symbfit{B} \right)^\top = \symbfit{B}^\top \symbfit{A}^\top$

\indent If $\symbfit{A}$ and $\symbfit{B}$ are both invertible:

\indent$\left( \symbfit{A}\symbfit{B} \right)^{-1} = \symbfit{B}^{-1}\symbfit{A}^{-1}$

\indent$\left( \symbfit{A}^{-1} \right)^\top = \left( \symbfit{A}^\top \right)^{-1}$
\end{document}